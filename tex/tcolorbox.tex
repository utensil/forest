\usepackage[amsmath,thmmarks]{ntheorem}
\usepackage[many]{tcolorbox}
% \tcbset
\tcbset{
	enhanced,breakable,
	titlerule=0mm,
	halign=left,parbox=false,
	theorem style=plain apart,
	left=2ex,right=2ex,boxsep=1.5ex,top=0mm,bottom=0mm,
	label separator=,
}
\newcounter{tcbthm}[section]

\newtcbtheorem[auto counter,number within=subsection]{definition}{Definition}{
	label type=definition,
	frame hidden,boxrule=0mm,sharp corners, % interior hidden,
	colbacktitle=white,
	colback=white,
	fonttitle=\bfseries\sffamily\color{black},
	fontlower=\normalfont,
}{}

\newtcbtheorem[use counter from=definition,crefname={convention}{conventions}]{convention}{Convention}{
	frame hidden,boxrule=0mm,sharp corners, % interior hidden,
	colbacktitle=white,
	colback=white,
	fonttitle=\bfseries\sffamily\color{black},
	fontlower=\normalfont,
}{}

\newtcbtheorem[use counter from=definition,crefname={notation}{notations}]{notation}{Notation}{
	frame hidden,boxrule=0mm,sharp corners, % interior hidden,
	colbacktitle=white,
	colback=white,
	fonttitle=\bfseries\sffamily\color{black},
	fontlower=\normalfont,
}{}

\newtcbtheorem[use counter from=definition]{remark}{Remark}{
	label type=remark,
	frame hidden,boxrule=0mm,sharp corners, % interior hidden,
	borderline west={2pt}{0mm}{ForestGreen},
	colbacktitle=ForestGreen!5,
	colback=ForestGreen!5,
	% colframe=ForestGreen,
	fonttitle=\bfseries\sffamily\color{ForestGreen!70!black},
	fontlower=\normalfont,
}{}

\newtcbtheorem[use counter from=definition]{example}{Example}{
	label type=example,
	boxrule=0.5pt,titlerule=0pt,titlerule style=Salmon!5,
	colbacktitle=Salmon!5,
	colback=Salmon!5,
	colframe=RawSienna,
	fonttitle=\bfseries\sffamily\color{RawSienna},
	fontlower=\normalfont,
}{}

\newtcbtheorem[use counter from=definition]{theorem}{Theorem}{
	label type=theorem,
	% same below
	frame hidden,boxrule=0mm,sharp corners, % interior hidden,
	borderline west={2pt}{0mm}{MidnightBlue},
	colbacktitle=TealBlue!5,
	colback=TealBlue!5,
	fonttitle=\bfseries\sffamily\color{MidnightBlue!70!black},
	fontlower=\normalfont,
}{}

\newtcbtheorem[use counter from=definition]{lemma}{Lemma}{
	label type=lemma,
	% same below
	frame hidden,boxrule=0mm,sharp corners, % interior hidden,
	borderline west={2pt}{0mm}{MidnightBlue},
	colbacktitle=TealBlue!5,
	colback=TealBlue!5,
	fonttitle=\bfseries\sffamily\color{MidnightBlue!70!black},
	fontlower=\normalfont,
}{}

\newtcbtheorem[use counter from=definition]{corollary}{Corollary}{
	label type=corollary,
	% same below
	frame hidden,boxrule=0mm,sharp corners, % interior hidden,
	borderline west={2pt}{0mm}{MidnightBlue},
	colbacktitle=TealBlue!5,
	colback=TealBlue!5,
	fonttitle=\bfseries\sffamily\color{MidnightBlue!70!black},
	fontlower=\normalfont,
}{}

\theoremstyle{nonumberbreak} % nonumberplain
% \theoremseparator{:\smallskip}
\theoremheaderfont{\bfseries\sffamily\color{MidnightBlue!70!black}}
% \theoremheaderfont{\itshape\rmfamily}
\theorembodyfont{\normalfont}
\theoremsymbol{\ensuremath{\square}}
\newtheorem{proof}{Proof}

\tcolorboxenvironment{proof}{
	enhanced,boxrule=0pt,frame hidden,sharp corners,
	top=0mm,bottom=0mm,left=0mm,right=0mm,
	colbacktitle=TealBlue!5,
	colback=TealBlue!5,
	after upper={\hfill\ensuremath{\square}}
}

