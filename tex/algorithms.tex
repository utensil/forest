% deps
\usepackage{amsmath,amsfonts}
\usepackage{graphicx}
% \usepackage{kvoptions}
% \usepackage{algorithmicx}
% \usepackage{etoolbox}
% \usepackage{fifo-stack}
% \usepackage{varwidth}
% \usepackage{tabto}
% \usepackage{totcount}
\usepackage{tikz}
\usetikzlibrary{calc,fit,tikzmark}
\usepackage{tabto}
\TabPositions{1.5cm} % Adjust as needed!

% algpseudocodex works as a drop-in replacement for algorithmic, with enhancements
\usepackage[endComment={\quad}]{algpseudocodex} % endComment for avoiding cropping the end of comment
% weirdly, the indent line is not working
\tikzset{algpxIndentLine/.style={draw=gray,very thin}}


% pseudo seems to be very straightforward
\usepackage[kw]{pseudo}
\pseudoset{label={\color{gray} \small\arabic*},ct-left=$\triangleright$} % indent-mark,line-height=0.5,

% algorithm somehow not working
% \usepackage{algorithm}
% instead, we use tcolorbox, adapted from doc of package pseudo
\usepackage{tcolorbox}
\tcbuselibrary{skins}
\tcbuselibrary{theorems}
\newtcbtheorem{algorithm}{Algorithm}{pseudo/ruled, float}{alg}

% fill out entire line, adapted from doc of package pseudo
\usepackage{tabularx}
\pseudodefinestyle{fullwidth}{
begin-tabular =
\tabularx{\linewidth}[t]{@{}
r % Labels
>{\pseudosetup} % Indent, font, ...
X % Code (flexible)
>{\leavevmode\small\color{gray}} % Comment styling
p{0.5\linewidth} % Comments (fixed)
@{}},
end-tabular = \endtabularx,
setup-append = \RestorePseudoEq
}

