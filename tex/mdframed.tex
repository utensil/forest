% theorem styles are implemented by package `mdframed` for LaTeX i.e. PDF
% mostly follows https://github.com/vEnhance/napkin/blob/main/tex/preamble.tex
\usepackage{amsthm}
% they have to be loaded afer `amsthm`
\usepackage{newpxtext,newpxmath}
\usepackage{thmtools}
\usepackage[framemethod=TikZ]{mdframed}

\theoremstyle{definition}
\mdfdefinestyle{mdbluebox}{%
	% same
	skipabove=4ex,
	skipbelow=4ex,
	innertopmargin=4ex,
	innerbottommargin=8pt,
	usetwoside=false,
	innermargin=2em,
	% tt-002B is too tall
	% nobreak=true,
	needspace=20em,
	frametitleaboveskip=5pt,
	frametitlebelowskip=0pt,
	% spaceabove=12pt,
	% spacebelow=1pt,
	% tt-002B is too tall
	% nobreak=true,
	needspace=10\baselineskip,
	frametitleaboveskip=5pt,
	frametitlebelowskip=0pt,
	% shared by left framed
	leftline=true,
	rightline=false,
	topline=false,
	bottomline=false,
	linewidth=2pt,
	% customizations
	linecolor=MidnightBlue,
	backgroundcolor=TealBlue!5,
}
\declaretheoremstyle[%
	% same
	headpunct={\\[3pt]},
	% customizations
	headfont=\sffamily\bfseries\color{MidnightBlue!70!black},
	mdframed={style=mdbluebox}
]{thmbluebox}

\mdfdefinestyle{mdredbox}{%
	% same
	skipabove=4ex,
	skipbelow=4ex,
	innertopmargin=4ex,
	innerbottommargin=8pt,
	usetwoside=false,
	innermargin=2em,
	% tt-002B is too tall
	% nobreak=true,
	needspace=20em,
	frametitleaboveskip=5pt,
	frametitlebelowskip=0pt,
	% spaceabove=12pt,
	% spacebelow=1pt,
	% tt-002B is too tall
	% nobreak=true,
	needspace=10\baselineskip,
	frametitleaboveskip=5pt,
	frametitlebelowskip=0pt,
	% shared by full framed
	linewidth=0.5pt,
	% customizations
	everyline=true,
	frametitlefont=\bfseries,
	linecolor=RawSienna,
	backgroundcolor=Salmon!5,
}
\declaretheoremstyle[
	% same
	headpunct={\\[3pt]},
	% customizations
	postheadspace={0pt},
	headfont=\bfseries\color{RawSienna},
	mdframed={style=mdredbox},
]{thmredbox}

\mdfdefinestyle{mdgreenbox}{%
	% same
	skipabove=4ex,
	skipbelow=4ex,
	innertopmargin=4ex,
	innerbottommargin=8pt,
	usetwoside=false,
	innermargin=2em,
	% tt-002B is too tall
	% nobreak=true,
	needspace=20em,
	frametitleaboveskip=5pt,
	frametitlebelowskip=0pt,
	% spaceabove=12pt,
	% spacebelow=1pt,
	% tt-002B is too tall
	% nobreak=true,
	needspace=10\baselineskip,
	frametitleaboveskip=5pt,
	frametitlebelowskip=0pt,
	% shared by left framed
	leftline=true,
	rightline=false,
	topline=false,
	bottomline=false,
	linewidth=2pt,
	% customizations
	linecolor=ForestGreen,
	backgroundcolor=ForestGreen!5,
}
\declaretheoremstyle[
	% same
	headpunct={\\[3pt]},
	% customizations
	headfont=\bfseries\sffamily\color{ForestGreen!70!black},
	bodyfont=\normalfont,
	mdframed={style=mdgreenbox},
]{thmgreenbox}

\declaretheoremstyle[
	% same
	headpunct={\\[3pt]},
	% customizations
	headfont=\bfseries\sffamily\color{ForestGreen!70!black},
	bodyfont=\normalfont,
	mdframed={style=mdgreenbox},
]{thmgreenbox*}

\mdfdefinestyle{mdblackbox}{%
	skipabove=4ex,
	linewidth=3pt,
	rightline=false,
	leftline=true,
	topline=false,
	bottomline=false,
	linecolor=black,
	backgroundcolor=RedViolet!5!gray!5,
}
\declaretheoremstyle[
	headfont=\bfseries\sffamily,
	bodyfont=\normalfont\small,
	spaceabove=0pt,
	spacebelow=0pt,
	mdframed={style=mdblackbox}
]{thmblackbox}

\declaretheoremstyle[
	headfont=\bfseries\sffamily,
	bodyfont=\normalfont, %\small, %\sffamily,
	spaceabove=12pt,
	spacebelow=1pt,
	headpunct={\\[3pt]},
]{def}

\numberwithin{equation}{subsection}
\theoremstyle{definition}
% \declaretheorem[style=thmblackbox,name=Definition,numberwithin=subsection]{definition}
\declaretheorem[style=thmbluebox,name=Theorem,numberwithin=subsection, numberlike=equation]{theorem}
\declaretheorem[style=thmbluebox,name=Theorem,numberwithin=subsection, numberlike=equation]{theoremx} % theorem not in dep graph
\declaretheorem[style=thmbluebox,name=Lemma,sibling=theorem]{lemma}
\declaretheorem[style=thmbluebox,name=Lemma,sibling=theorem]{lemmax} % lemma not in dep graph
\declaretheorem[style=thmbluebox,name=Corollary,sibling=theorem]{corollary}
\declaretheorem[style=thmbluebox,name=Proposition,sibling=theorem]{proposition}

\theoremstyle{definition}
% \newtheorem{claim}[theorem]{Claim}
% \newtheorem{definition}[theorem]{Definition}
\declaretheorem[style=def,name=Definition,sibling=theorem]{definition}
\declaretheorem[style=def,name=Convention,sibling=theorem]{convention}
\declaretheorem[style=def,name=Notation,sibling=theorem]{notation}
% \newtheorem{fact}[theorem]{Fact}
% \newtheorem{abuse}[theorem]{Abuse of Notation}

\declaretheorem[style=thmredbox,name=Example,sibling=theorem]{example}

\theoremstyle{theorem}
% \declaretheorem[name=Question,sibling=theorem,style=thmblackbox]{ques}
% \declaretheorem[name=Exercise,sibling=theorem,style=thmblackbox]{exercise}

\declaretheorem[name=Remark,sibling=theorem,style=thmgreenbox]{remark}
\declaretheorem[name=Remark,sibling=theorem,style=thmgreenbox*]{remark*}
% \declaretheorem[name=Step,style=thmgreenbox]{step} % only used in Lebesgue int

\crefname{notation}{notation}{notations}
\crefname{convention}{convention}{conventions}